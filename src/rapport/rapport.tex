\documentclass{article}
\begin{document}
\title{Projet ASPP3}
\author{Charles Le Mero\\Louis Douriez\\Loris Lucido}
\tableofcontents
\section{Lien du projet}

Vous trouverez dans les sources du projet, un fichier \textbf{README.md} qui contient les instructions de compilation, d'exécution et de test de notre programme.
\\

Le dépôt est lui disponible en copiant la commande suivante :
\\
\begin{verbatim}
git clone https://github.com/Novskilol/ASPP3
\end{verbatim}

\section{Premiers pas avec grammar.y}
En ce qui concerne la grammaire, nous n'étions pas du tout certains de son fonctionnement, cela nous a pris du temps pour choisir la façon d'implémenter les balises, notamment pour la surbrillance et les définitions des fonctions et des variables, nous avons d'abord pensé a du javascript qui parserai tout le fichier rempli de balises, ensuite nous avons implémenté un postlex qui était appliqué après le premier yyparse pour transformer les marqueurs. Finalement Marc Zeitoun nous a expliqué plus précisement le fonctionnement de la grammaire bison et nous avons préféré cette solution.\\

\section{Code C }

\subsection{Indentation et mise en page}

Après avoir compilé et observé le fonctionnement du parseur fourni par le sujet nous nous sommes lancé dans l'indentation et la mise en page du code C.
\\

Notre première approche par rapport à l'indentation fut de modifier le fichier lex pour rajouter des balises NEWLINE en fin de point-virgule et en fin de ligne (en ayant au préalable supprimer par défaut tous les sauts de ligne).
\\

De plus à chaque fois que l'on rencontre un \{ ou un \} on modifie une variable représentant le niveau d'indentation. Ce fonctionnement posait certains problèmes dans des cas plus complexes comme par exemple un \textit{if} suivi d'une instruction mais sans crochet . Nous avons donc décidé de reporter le travail d'indentation ainsi que de mise en page sur le fichier bison.
\\

Ainsi nous avons pu traiter les cas plus complexes de mises en pages. Nous nous sommes donc occupés des cas particuliers comme la présence de point-virgule dans les boucles \textit{for}.

\subsection{Code expand collapse}

Le code expand/collaspe est géré entièrement en jQuery.
\\

Chaque accolade rencontrée dans le bison est disposée dans un tag \textit{<braces>}. Un tag \textit{<block>} est placé avant et après le bloc de code pour délimiter la zone à cacher/montrer.

\begin{verbatim}
<block>
	<braces>{</braces>
		<item>
			<block>
				// CONTENT
			</block>
		</item>
	}
</block>

\end{verbatim}

\subsection{Highlight et goto declaration}

Chaque déclaration de variable ou de fonction est placé dans un tag \textit{<declaration>} tandis que chaque variable ou appel de fonction est placé dans un tag \textit{<identifier>}.
\\

Pour chaque déclaration, un indentifiant unique est choisi puis une classe portant cet identifiant est donnée à la balise html.

\begin{verbatim}
int <declaration class="1">function</declaration>(){ }

<identifier class ="1">function</identifier>();

\end{verbatim}

Pour lier une variable et sa déclaration, il suffit de donner le même numéro de classe à la déclaration et à toutes les occurences de la variable. Le code jQuery se charge de mettre les surbrillance toutes les occurences d'une même variable ou de déplacer la page vers la déclaration d'une variable.
\\

Il faut faire attention aux niveaux d'indentation (i.e. dans quel bloc de code je me trouve) avant de déterminer quelle classe donner à ma déclaration ou ma variable. C'est ici qu'intervient la structure \textit{symbolTable}.
\\

Il s'agit d'une pile où l'on insère des listes de symboles déclarés dans le code. Chaque niveau de la pile correspond à un niveau d'indentation.
\\
On augmente notre niveau de pile à chaque ouverture d'accolade mais aussi lorsque l'on déclare les arguments d'une fonction.
\\

En effet, les variables globales d'un fichier étant de niveau de pile 0, on ne peut pas avoir un même niveau de pile pour les arguments de fonctions.

\begin{verbatim}
int i; // variable globale, niveau 0
int function (int i, int j){ // argument de fonction, niveau 1
    {
        int i; // déclaration dans un bloc dans une fonction, niveau 3
    }
}

\end{verbatim}

On diminue d'un niveau à chaque fermeture d'accolades et à chaque fois que l'on a terminé d'écrire les arguments d'une fonction.

A chaque fois que l'on rencontre une déclaration, on la rajoute à la liste du niveau de pile adéquat. Il faut cependant différencier le cas où l'on rencontre la déclaration d'une fonction  du cas où l'on rencontre sa définition.
\\

Pour cela, à chaque déclaration, on regarde dans la liste se trouvant en top de pile s'il n'existe pas déjà un \textit{identifier} avec le même nom. Si oui on lui donne alors la même classe, sinon on l'ajoute.

\subsection{Gestion des types}

Pour que notre programme fonctionne sur nos propres fichiers il a fallut prendre en compte les déclarations et définitions de type : \textit{typedef} ou \textit{struct}. Pour ce faire nous avons créé une liste de types contenant l'ensemble des types partagés entre tous nos fichiers.
\\

Ainsi nous avons une liste de types commun à tous nos fichiers. Cependant les types dénifis dans des librairies externes (exemple : \textit{bool}, \textit{size_t}, \textit{FILE}...) ne sont pas reconnu par défaut. Nous avons ajouté une règle à flex pour accepter en plus ces trois types que nous utilisons dans nos fichiers.

\subsection{Affichage prototype et documentation}

\subsubsection{Parse des balises doxygen}

Pour pouvoir utiliser des commentaires type doxygen à la fois dans les \textit{tooltip} et dans la page de documentation des fonctions, nous avons créé une structure dédié : \textit{functionParser}.
\\

A chaque fois que le fichier lex lit un ensemble de balises de documentation, il l'enregistre dans \textit{fonctionParser}.
\\

Une fois que bison arrive en fin de déclaration ou de définition, il parse le contenue de \textit{functionParser} et affiche dans un fichier \textit{nom_source.doc.html} tout le code html correspondant. Puis on vide \textit{fonctionParser}.
\\

La structure \textit{fonctionParser} est modulable puisqu'elle contient des règles de parsing qui sont activés lorsqu'un motif spécifique est rencontré .
\\
Chaque règle se voit offert la possibilité d'écrire dans un fichier correspondant à la documentation et au tooltip ( de manière indifférenciée).
\\

 Au niveau de la consigne nous pensions qu'il s'agissait de parser les commentaires indépendamment du parse latex. Nous nous sommes rendu compte 1 jour avant le rendu écrit qu'il s'agissait d'utiliser le parseur latex pour la documentation.
 \\
 Ceci n'a donc pas été implémenté.


\subsubsection{Prototype}

Chaque type rencontré (ainsi que les '*' désignant un type pointeur) est sauvegardé dans une variable, il sera donné plus tard à \textit{functionParser}.
\\
Pour les fonctions, on stocke tous les caractères rencontrés (les arguments de la fonction) jusqu'à ce qu'on arrive à la fin du prototype de la fonction puis on appelle \textit{functionParser}.

\subsubsection{Tooltip}

Le Javascript se charge de créer un calque pour chaque \textit{identifier} à partir de son attribut html \textit{title}.
\\
\begin{verbatim}
<identifier title='ici se trouve le contenu du tooltip' class="1">main</identifier>
\end{verbatim}
Pour donner un calque à une fonction, on doit parcourir toutes les balises doxygen, ainsi que l'intégralité du prototype de la fonction.
\\
On crée ensuite une balise \textit{<docuForTooltip>} vide ayant la même classe que la dernière fonction rencontrée (celle dont on cherche à générer un calque) et insère la documentation dans le \textit{title} de la balise.

\begin{verbatim}
int // le type de retour
	<declaration class='1'>
		main
	</declaration>
(int argc, char ** argv) // le prototype
	<docuForTooltip title='ici se trouve la documentation parsée' class='1'>
	</docuForTooltip>
\end{verbatim}
S'il n'existe pas de documentation pour une fonction, on utilise uniquement le prototype.
\\

Un travail similaire est aussi effectué pour les variables, on peut documenter une variable.
\\

Pour lier la balise vide \textit{<docuForTooltip>} à l'\textit{identifier}, on utilise jQuery qui va rechercher toutes ces balises, et donner le meme titre à chaque \textit{identifier} ayant la meme classe.
\\

Le problème étant de garder en mémoire la classe de la dernière fonction rencontrée.
 Pour résoudre ce problème ainsi que pour implémenter les tooltip multi-fichiers (i.e. une fonction documentée dans une header se voit toute ses occurrences agrémentées d'un tooltip portant la même documentation) nous avons identifié quelques restrictions afin que tout fonctionne bien, à savoir :
\begin{enumerate}
\item Une même fonction ne peut être documentée qu'un seule fois, dans sa définition ou déclaration, dans un source C ou un header
\item La grammaire fournie ne permet pas les déclarations de fonctions à l'intérieur même d'autres fonctions. Par exemple le cas ci-dessous provoque une erreur de syntaxe.

\begin{verbatim}
void f() {
	void g() {

	}
	g();
}
\end{verbatim}
\item Une déclaration et une définition de fonction sont toutes deux considérées par la grammaire comme une déclaration d'un \textit{identifier}. De plus on ne peut pas différencier une déclaration de variable d'une déclaration de fonction.
\\
Cela pose plusieurs problèmes car nous avons besoin de différencier ces deux cas pour récupérer le bon numéro de classe (celui de la dernière fonction déclarée).
\\
En considérant le point abordé précédemment, on peut dire qu'aucune déclaration ayant le même nom se trouve au même niveau d'indentation.
\\
Il suffit alors de stocker la classe d'un \textit{identifier} uniquement au niveau d'indentation (ou niveau de pile) 0.
\end{enumerate}

\section{Latex}
Pour réaliser la partie latex, nous avons choisit d'utiliser un flex et un bison, au début nous faisions tout dans le flex mais nous nous sommes vites rendus compte que cela allait poser de gros problèmes et qu'une grammaire serait très pratique.\\

\subsection{Sauts de lignes et begin}
L'implémentation des sauts de lignes / des espaces n'ont pas posé de problèmes particuliers, en revanche begin et end ont posé la question des environnements et de graves problèmes de conflits ont commencés à s'immisser dans la grammaire sans que nous ne nous en rendions compte.L'implementation des differents types \textit{ITEM,ABSTRACT,ENUMERATE,EQUATION,EQUATION*,} posa de nombreux problèmes: \\
\begin{itemize}
\item ABSTRACT : écrire du texte centré, et il pose aussi la question de pouvoir utiliser le mot abstract ainsi que les autres mots clés en dehors d'un \backslashbegin
\item ITEM : item a posé la question des petits points, vite résolue, énumerate et itemize ont été traités de façon similaire
\item EQUATION : le fait d'utiliser les labels pour numéroter les équations qui doivent être centrées et le nombre qui doit être a droite a posé des questions par rapport aux balises div : exemple
\begin{equation}
 x+y=5
\end{equation}
\end{itemize}
\subsection{Titre et auteurs}
Ces deux commandes ont posé quelques problèmes de centrage.
\\
\subsection{Tabular}
Tabular ne fut pas une partie de plaisir, le fait de compter le nombre de 'l' 'c' et 'r' posa des problèmes de grammaire, puis l'implémentation d'un tableau en html faisait qu'il y avait toujours un problème par rapport a la dernière case du tableau du au fait qu'il était impossible de savoir si on allait rencontrer une nouvelle ligne ou non.

\\Un verrou nous a finalement permis de régler ce petit problème. De plus tabular ajouta un problème en tant que titre de subsection (non prévu dans la grammaire -> grammaire modifiée).
\\
\subsection{Conflits}
Le projet avancant sans problème de compilation et n'ayant pas trop prêté attention aux warnings, les conflits sont devenus de plus en plus nombreux sournoisement jusqu'à atteindre le nombre record de 284 conflits S/R et 84 conflits R/R qu'il a fallut débugger.

\\Après analyse du fichier \textbf{latex.output} puis une reconstruction minutieuse de la grammaire règle par règle il a suffit de changer une règle pour la rendre non récursive droite et non effaçable pour effacer l'intégralité des conflits d'un seul coup.
\\
\subsection{Le cas Verbatim}
Après de longues reflexions par rapport a verbatim, ce que nous avons choisit de faire fut de ne pas toucher à ce qui se trouve a l'intérieur des balises, ainsi il a fallut modifier le fichier latex.l afin d'y ajouter un système qui permet de laisser passer tout les caractères sauf le \backslashend ce qui provoqua des erreurs de syntaxe.

\\Après une mise en place d'un système de verrous nous avons réussit a maitriser le problème en utilisant deux nouveaux modes et en traitant la parenthèse \} suivant le verbatim séparement.
\\
\subsection{Label et Ref}
label et ref ont posé la question du déplacement via clic dans le fichier, nous avons utilisé du code jQuery et Javascript pour nous déplacer dans la documentation des fichiers c, nous avons donc réutilisé le code déja implémenté pour label et ref, les espaces et les : dans les accolades ont posé problème nous les avons donc supprimés.
\\
\subsection{Table of contents}
La table des contenus fut difficile à implémenter, deja il fallait mettre des balises pour que les sections dans la table nous amènent directement aux sections réelles, ensuite il a fallut écrire à chaque section / sous section croisée a la fois dans le fichier rapport et dans un autre qui sert de table de contenus, puis au final a l'aide de plusieurs dup2 fopen et fflush venir greffer la table des contenus au bout du rapport.
\\
\subsection{Des begin dans des begin}
Nous nous sommes rendus compte trop tard que ceci pouvait poser problème, nous avons donc choisit une stack permettant de se remettre au mode précédent a chaque mot clé rencontré puis passé.

\\En fait, tout le texte rencontré est modifié en fonction du mode courant, il est donc essentiel de se rapeller des modes passés pour les remettre ensuite, mais cette solution n'est pas optimale et mériterais d'être améliorée.
\\
\section{Debuggage}

Au fur et à mesure de l'avancement du projet nous avons fait attention à ne pas avoir de fuites mémoires. Nous avons donc utilisé valgrind pour supprimer les erreurs et les fuites mémoires.

\section{Améliorations possibles}

Le projet final fonctionne assez bien sur des fichiers sources C et les headers, cependant de nombreuses améliorations sont possibles notamment :
\\
\begin{itemize}
\item Pouvoir cliquer sur les définitions de fonctions pour allez sur la documentation (on peut pour le moment cliquer sur les déclarations pour revenir au niveau de la définition dans un même fichier)
\item Permettre d'aller au fichier contenant la définition d'une fonction en cliquant sur son nom.
\item Parser les règles des fichiers bisons et lex et afficher la documentation.
\item Donner la documentation doxygen à parser à notre parseur latex
\item Donner les bouts de code C présents dans le document latex à parser à notre parseur C
\item Ainsi que tout ce qui a été demandé dans le sujet et non réalisé (par manque de temps).
\item Faire ce qui concerne mathML,citecode et les extensions
\item Traiter les arguments optionnels de begin
\item Retravailler la stack des empilements des états
\end{itemize}


\end{document}
