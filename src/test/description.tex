% Tout ce qui est compris entre le caractère % et une fin de ligne est un
% commentaire ignoré par LaTeX

% Un fichier LaTeX commence par une commande \documentclass qui déclare le type
% de document

\documentclass{article}

% Ici, et jusqu'au \begin{document} ci-dessous, c'est le préambule : on y met
% les déclarations de style, les définitions de macros, etc.

%%%% Personnalisation du style défaut

% Fichiers de style utilisés ; les déclarations se font au moyen de la commande
% \usepackage

\usepackage{fullpage} % Agrandit les dimensions du texte (hauteur, largeur,
                      % etc.) par rapport à celles par défaut. Attention
                      % ce package ne se trouve pas dans toutes les
                      % distributions LaTeX

\usepackage[french]{babel} % Pour adopter les règles de typographie française

\usepackage[utf8]{inputenc} % Prévenir LaTeX de l'encodage

\usepackage{amsmath} % Les bibliothèques LaTeX de l'American
                     % Mathematical Society sont pleines de macros
                     % intéressantes (voire indispensables).

%%%% Déclaration d'environnements

% Les définitions, théorèmes et autres lemmes, corollaires, propositions,
% exemples, se déclarent au moyen de \newtheorem

\newtheorem{de}{Définition}[subsection] % les définitions et les théorèmes sont
\newtheorem{theo}{Théorème}[section]    % numérotés par section
\newtheorem{prop}[theo]{Proposition}    % Les propositions ont le même compteur
                                        % que les théorèmes
