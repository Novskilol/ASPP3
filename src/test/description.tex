% this is a comment please ignore Kappa Kappa

\documentclass{article}
\usepackage{graphicx}
\usepackage{amsmath}

\begin{document}

\title{jambon}

\author{Author's Name,\\
   Andrew Roberts,\\
   école d'informatique
}

\tableofcontents
\maketitle

\begin{abstract}
  \textbf{The}.
  \textit{abstracttextgoeshere}
  
  and there is more just to know how it goes, tis the story of the long long johnson,
  oh don piano where have you been my dear lord?
\end{abstract}

\section{test}
Here is the

text of your \ introduction.

\begin{tabular}{llc}
Case 1,1&Case 1,2&Case 1,3\\
Case 2,1&Case 2,2&Case 2,3\\
\end{tabular}

\begin{enumerate}
\item one
  Test
\item deux
\item trois
\end{enumerate}

\begin{itemize}
\item one
  Test
\item deux
\item trois
\end{itemize}

\begin{equation} 
\label{eq:test}
     y = 5
\end{equation}

\begin{equation*}
  x = 2
\end{equation*}

\begin{equation}
     y = 10
\end{equation}

\begin{verbatim}
assert(for (int i=0;i<50;i++)break;)

\end{verbatim}

\subsection{Subsection Heading Here}
Après avoir compilé et observé le fonctionnement du parseur fourni par le sujet nous nous sommes lancé dans l'indentation et la mise en page du code C . Notre première a
\subsubsection{Well hello there}
 Write your subsubsection text here.
Write your subsection text here.
\subsection{Subsection Heading Here 2 boys}
Ici la subsection number two.
\section{Conclusion}
Write your conclusion here.
\ref{eq:test}{preuve}
\end{document}
